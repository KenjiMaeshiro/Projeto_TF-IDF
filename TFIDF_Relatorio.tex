\documentclass[12pt]{article}
\usepackage[utf8]{inputenc}
\usepackage[T1]{fontenc}
\usepackage[brazil]{babel}
\usepackage{geometry}
\usepackage{graphicx}
\usepackage{booktabs}
\usepackage{setspace}
\usepackage{float}
\graphicspath{{Prints/}}
\geometry{a4paper, margin=2.5cm}
\setstretch{1.3}

\begin{document}

\begin{center}
\textbf{\Large FACULDADE DE TECNOLOGIA DA BAIXADA SANTISTA – FATEC RUBENS LARA}\\[0.5cm]
\textbf{\large CURSO DE CIÊNCIA DE DADOS}\\[3cm]

\textbf{\LARGE ANÁLISE DE REVIEWS DA STEAM UTILIZANDO TF-IDF}\\[3cm]

\textbf{Aluno:} Caio Kenji de Paula Maeshiro\\
\textbf{Disciplina:} Álgebra Linear\\
\textbf{Período:} 2º semestre de 2025\\[3cm]

Santos – SP\\
2025
\end{center}

\newpage
\section*{Descrição do Dataset}

O projeto tem como o objetivo a análise de um dataset composto por avaliações públicas da plataforma Steam. O dataset utilizado contém mais 6,4 milhões de reviews em inglês, disponibilizados de forma pública e anônima, reunindo informações sobre a opinião de usuários a respeito de diversos jogos disponíveis na plataforma.  

Cada registro do conjunto de dados inclui as seguintes colunas:

\begin{itemize}
    \item \textbf{Review text:} o texto livre da avaliação feita pelo usuário;
    \item \textbf{Game ID:} o identificador numérico do jogo ao qual a avaliação pertence;
    \item \textbf{Sentiment:} o sentimento da review, podendo ser classificado como positivo ou negativo;
    \item \textbf{Helpful:} número de usuários que consideraram aquela avaliação útil.
\end{itemize}

O arquivo original foi disponibilizado em formato CSV compactado. Devido ao seu grande volume de informações, totalizando mais 6 milhões de registros. Para fins de processamento e viabilidade computacional, foi considerado apenas uma amostra de 100.000 de registros, extraída a partir do dataset completo. A extração foi feita no script chamado “Filtro” e a amostra foi salva em um novo arquivo denominado “reviews\_filtradas.csv”, que serviu de base para as etapas seguintes. 

A partir dessa amostra, foi realizado um filtro específico para o jogo de ID 10180, que corresponde ao título Call of Duty: Modern Warfare 2 (2009). Essa filtragem teve como finalidade concentrar o estudo em um único jogo bastante avaliado pelos usuários. 

\section*{Tema do Projeto}

Este trabalho se dedica fazer a análise de avaliações sobre o jogo Call of Duty: Modern Warfare 2 (2009), buscando descobrir as semelhanças entre a primeira avaliação registrada e as subsequentes. Para alcançar este objetivo, empregar-se o método TF-IDF (Term Frequency – Inverse Document Frequency), que transforma textos em dados numéricos com base na importância das palavras.

Com essa representação vetorial, foi possível calcular o grau de similaridade entre as avaliações por meio da métrica de similaridade do cosseno, identificando quais reviews apresentam conteúdos mais próximos entre si. 

\section*{Etapas Realizadas}

\begin{figure}[H]
    \centering
    \includegraphics[width=0.9\textwidth]{print1}
    \label{fig:print1}
\end{figure}

Inicialmente, realizou-se a redução do tamanho do conjunto de dados original para facilitar a análise. O arquivo original, lido através da biblioteca Pandas, era extenso, então optou-se por pegar uma amostra, pegando as primeiras 100.000 linhas. Essa seleção foi armazenada em um novo arquivo, nomeado "reviews\_filtradas.csv", que serviu como base para as fases seguintes do processo.

\begin{figure}[H]
    \centering
    \includegraphics[width=0.9\textwidth]{print2}
    \label{fig:print2}
\end{figure}

Na sequência, os dados do arquivo passaram por um processo de leitura, limpeza e filtragem. Primeira, algumas colunas que não agregava ao projeto foram descartadas, enquanto outras receberam novos nomes para facilitar o entendimento do conjunto de dados. Em seguida, aplicou-se um filtro para isolar os dados do jogo com ID 10180, correspondente ao título Call of Duty: Modern Warfare 2 (2009), focando a análise nas opiniões sobre esse título específico.

Após o filtro, todas as avaliações foram convertidas para o tipo texto, garantindo consistência na manipulação dos dados. Também foram removidas eventuais duplicatas, de modo a evitar que repetições influenciassem nos resultados. Por fim, o índice do conjunto de dados foi redefinido, resultando em um DataFrame limpo.

\begin{figure}[H]
    \centering
    \includegraphics[width=0.9\textwidth]{print3}
    \label{fig:print3}
\end{figure}

Depois da filtragem de dados, foi realizado o tratamento do texto, buscando uniformizar e aprontar as avaliações para a análise quantitativa.

Inicialmente, as stopwords da biblioteca NLTK (Natural Language Toolkit), que são palavras muito comuns no inglês, foram carregadas e ajustadas. Essas palavras foram removidas para que a análise se concentrasse nas palavras-chave das opiniões dos usuários.

Em seguida, foi criada a função "limpar\_review". Essa função executa uma série de transformações:

\begin{itemize}
    \item \textbf{Converte todo o texto para letras minúsculas} garantindo uniformidade entre palavras semelhantes escritas de maneiras diferentes;
    \item \textbf{Remove símbolos e caracteres especiais} como pontuações, números e códigos HTML;;
    \item \textbf{Divide o texto em palavras isoladas} possibilitando o tratamento individual de cada termo;
    \item \textbf{Elimina as stopwords} mantendo apenas as palavras de maior relevância semântica;
    \item \textbf{Reagrupa o texto limpo} em uma nova forma padronizada.
\end{itemize}

O resultado desse processo foi colocado em uma nova coluna chamada Review\_limpa, que contém a versão processada de cada avaliação original. Essa coluna serviu de base para as etapas seguintes de vetorização e cálculo de similaridade.

\begin{figure}[H] 
    \centering
    \includegraphics[width=0.9\textwidth]{print4}
    \label{fig:print4}
\end{figure}

Com os textos já limpos, foi aplicada a técnica TF-IDF. A implementação foi feita utilizando o recurso TfidfVectorizer() da biblioteca scikit-learn, resultando em uma matriz esparsa, designada como reviews\_matrix.

A seguir, empregou-se a função cosine\_similarity() para quantificar o nível de similaridade entre as diversas análises, com foco na primeira avaliação como a base de comparação. 

Os resultados foram então colocados em ordem decrescente, o que facilitou a identificação de avaliações com conteúdo bastante similar à avaliação inicial, revelando tendências e assuntos recorrentes nas impressões dos jogadores.

\begin{figure}[H] 
    \centering
    \includegraphics[width=0.9\textwidth]{print5}
    \label{fig:print5}
\end{figure}

Por fim, o código exibe no terminal a review base utilizada como referência e as dez avaliações mais semelhantes encontradas pelo cálculo de similaridade do cosseno. O laço for percorre os índices e valores de similaridade armazenados na lista similaridades, apresentando para cada item o texto correspondente e o respectivo grau de semelhança numérica.

\section*{Análise dos Resultados}

Após empregar o método TF-IDF, calculou-se a similaridade de cosseno entre as opiniões sobre o jogo Call of Duty: Modern Warfare 2 (2009). A intenção era descobrir quais análises mostravam uma ligação semântica mais forte com a primeira análise (posição 0).

A avaliação base selecionada pelo código foi: 

\begin{quote}
\textbf{Base Review 0:}\\
``makarov thought saw ghost got spooked dropped soap soon paid price''
\end{quote}

\subsection*{Resultado das 10 avaliações mais semelhantes}

\begin{center}
\begin{tabular}{cccc}
\toprule
\textbf{Posição} & \textbf{Índice} & \textbf{Trecho da Review} & \textbf{Similaridade} \\
\midrule
1 & 2961 & ive played almost every call duty title... selling soap... & 0.24 \\
2 & 1492 & revenge ghost roach die soap price kill shepherd lol & 0.24 \\
3 & 2681 & better ghost & 0.23 \\
4 & 1759 & dont drop soap price decrease & 0.20 \\
5 & 1129 & rip ghost & 0.19 \\
6 & 3416 & got & 0.18 \\
7 & 2581 & great game rip ghost & 0.18 \\
8 & 4731 & people used play ghost game & 0.18 \\
9 & 3655 & dont drop soap & 0.17 \\
10 & 5065 & bought played thing saw multiplayer explosions... & 0.17 \\
\bottomrule
\end{tabular}
\end{center}

\section*{Interpretação das Similaridades}

Os valores de similaridade variam entre 0.17 e 0.24, indicando um nível moderado de proximidade semântica, que já era esperado, pois as reviews são curtas, escritas de forma informal e com vocabulário variado. Assim, faz com que o TF-IDF encontre poucas palavras idênticas entre os textos.

O TF-IDF atribui maior relevância às palavras que aparecem com frequência no documento. Dessa forma, as palavras como “ghost”, “soap”, “price” e “makarov” adquiriram grande relevância e tiveram um impacto significativo nas semelhanças identificadas.

As avaliações mais próximas exibem uma grande concordância nos principais termos, além de citarem personagens e eventos semelhantes do jogos, o que justifica a similaridade próxima de 0.24.

Abaixo estão os ângulos das 10 reviews com o maior cálculo da similaridade do cosseno: 

\subsection*{Ângulos correspondentes às similaridades}

\begin{center}
\begin{tabular}{cc}
\toprule
\textbf{Similaridade} & \textbf{Ângulo aproximado} \\
\midrule
0.24 & 76,1° \\
0.23 & 76,7° \\
0.20 & 78,5° \\
0.19 & 79,0° \\
0.18 & 79,6° \\
0.17 & 80,2° \\
\bottomrule
\end{tabular}
\end{center}

Os resultados mostram que as reviews não são idênticas, mas compartilham informações e termos do jogo em comum,  principalmente os personagens “Ghost” e “Soap”. 

\section*{Conclusão}

O projeto demonstrou a eficácia do uso combinado das técnicas TF-IDF e similaridade de cosseno para identificar as similaridades das avaliações de usuários ao jogo Call of Duty Modern Warfare 2 (2009).

A análise mostrou que é possível detectar termos comuns e padrões de linguagem entre diferentes review. Mesmo sendo avaliações curtas e informais.
\end{document}
